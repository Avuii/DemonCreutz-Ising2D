\documentclass[12pt,a4paper]{article}
\usepackage[polish]{babel}
\usepackage[utf8]{inputenc}
\usepackage[T1]{fontenc}
\usepackage{graphicx}
\usepackage{float}
\usepackage{anyfontsize}
\usepackage{geometry}
\usepackage{placeins}
\usepackage{setspace}
\usepackage{amsmath}
\usepackage{titlesec}
\usepackage{lmodern}
\usepackage{enumitem}
\setlist[description]{style=nextline, font=\normalfont, labelwidth=6.2cm, leftmargin=6.4cm, itemsep=2pt}
\geometry{top=1.5cm, bottom=2.2cm, left=2.2cm, right=2.2cm}
\setstretch{1.3}
\usepackage{enumitem}
% Ustawienia globalne dla list
\setlist[itemize]{noitemsep, topsep=0pt, parsep=0pt, partopsep=0pt, leftmargin=1.2cm}
\setlist[enumerate]{noitemsep, topsep=0pt, parsep=0pt, partopsep=0pt, leftmargin=1.2cm}
\titlespacing*{\section}{0pt}{10pt}{5pt}
\titlespacing*{\subsection}{0pt}{8pt}{3pt}


\begin{document}

\begin{titlepage}
    \centering
    \vspace*{1cm}

    {\LARGE \textbf{Uniwersytet Łódzki}}\\[0.3cm]
    {\Large \textbf{Wydział Fizyki i Informatyki Stosowanej}}\\[2cm]

    {\normalsize \textbf{SPRAWOZDANIE Z LABORATORIUM}}\\[0.1cm]
    {\Large \textbf{Zaawansowane metody obliczeniowe}}\\[4cm]

{\fontsize {30pt}{20pt} \textbf{Demon Creutza}} \\[0.5cm] 
{\fontsize{25pt}{20pt}\textbf {w dwuwymiarowym modelu Isinga}}\\[2cm]


    \vfill

    \begin{flushright}
        \begin{tabular}{rl}
            \textbf{Autor:} & Katarzyna Stańczyk \\[0cm]
            \textbf{Kierunek studiów:} & Informatyka \\[0cm]
            \textbf{Data wykonania:} & 6 listopada 2025 \\[0cm]
            \textbf{Rok akademicki:} & 2025/2026 \\
        \end{tabular}
    \end{flushright}

    \vspace{2cm}

    {\large Łódź, 2025}

\end{titlepage}
\vspace*{0.5cm}
\section{Wstęp}
Celem ćwiczenia było zbadanie działania algorytmu Creutza w dwuwymiarowym modelu Isinga z wykorzystaniem metody Monte Carlo~\cite{Gwizdalla2010,Brandt1998}, a także otrzymanie wykresu zależności średniej magnetyzacji~$\langle m \rangle$ od temperatury~$T$ i porównanie go z wartościami teoretycznymi.

W ramach symulacji analizowano przebieg zmian Energii demona $E_D$, średnią magnetyzację~$\langle m \rangle$ na dany krok Monte Carlo oraz zależność średniej magnetyzacji~$\langle m \rangle$ od obliczonej temperatury~$T$, co pozwoliło ocenić poprawność działania algorytmu i jego zdolność do odtworzenia przejścia fazowego w modelu Isinga.\\


\subsection{Model Isinga}
Model Isinga stanowi jedno z podstawowych narzędzi w fizyce statystycznej i służy do opisu zjawisk magnetycznych, takich jak przejście między stanem uporządkowanym (ferromagnetycznym) a nieuporządkowanym (paramagnetycznym).

W rozpatrywanym przypadku spiny $s_i = \pm 1$ są rozmieszczone w punktach dwuwymiarowej siatki, a energia układu wyraża się zależnością:
\begin{equation}
E = -J \sum_{\langle i,j \rangle} s_i s_j,
\label{eq:ising_energy}
\end{equation}
gdzie 
\begin{itemize}
\item suma przebiega po parach sąsiednich spinów $\langle i,j\rangle$, 
\item  $J>0$ dla układu ferromagnetycznego.
\end{itemize}
Podstawową wielkością charakteryzującą uporządkowanie układu jest magnetyzacja, definiowana jako średnia wartość wszystkich spinów:
\begin{equation}
m = \frac{1}{N}\sum_{i=1}^{N} s_i,
\label{eq:magnetization}
\end{equation}
gdzie \(N = L_x \times L_y\) oznacza całkowitą liczbę spinów w siatce.\\ 
W niskiej temperaturze siatka jest uporządkowana (duża magnetyzacja $\langle m\rangle$), natomiast powyżej temperatury krytycznej uporządkowanie zanika.\\ 
Dla modelu dwuwymiarowego wartość tej temperatury wynosi \cite{Gwizdalla2010}:
\begin{equation}
\label{eq:Tc_teor}
T_c^{\mathrm{teor}}\approx 2.269\frac{J}{k_B}.
\end{equation}
\newpage
\subsection{Metoda Monte Carlo i demon Creutza}
Metoda Monte Carlo pozwala symulować procesy statystyczne, dla których uzyskanie rozwiązania analitycznego jest trudne lub niemożliwe.  
Demon Creutza to jej wariant, wprowadza się w nim dodatkowy element — tzw. demona, który pełni rolę pomocniczego „magazynu” energii $E_d$.  
Demon może przekazywać lub odbierać energię od układu, dzięki czemu całkowita energia (układ + demon) pozostaje stała. Zasada działania tego mechanizmu wymiany energii została szczegółowo opisana w kolejnym podrozdziale.

Prawdopodobieństwo wystąpienia stanu o danej energii $E_d$ opisuje rozkład Gibbsa\cite{Gwizdalla2010}:
{ 
  \setlength{\abovedisplayskip}{6pt}   
  \setlength{\belowdisplayskip}{6pt}   
  \begin{equation}
    P(E_d)\propto e^{-E_d/(k_B T)},
    \label{eq:demon_distribution}
  \end{equation}
}
1
\noindent gdzie \(k_B\) - stała Boltzmanna, \(\; T\) — temperatura układu.\\
Zatem wykres $\ln P(E_d)$ względem $E_d$ powinien być liniowy. Po przekształceniu równania otrzymujemy zależność liniową:
\begin{equation}
\ln P(E_d)=a E_d+b, \qquad a=-\frac{1}{k_B T}.
\label{eq:regression}
\end{equation}
Dlatego temperaturę można wyznaczyć na podstawie współczynnika kierunkowego \(a\):
\begin{equation}
\label{eq:temperature}
T=-\frac{1}{a k_B}.
\end{equation}

\subsection{Algorytm symulacji}
W każdej iteracji losowo wybierany jest spin $s_i$; następnie liczymy zmianę energii $\Delta E$ po odwróceniu spinu $s_i$:
\begin{equation}
\Delta E = 2 J s_i \sum_{\text{nn}} s_j,
\label{eq:deltaE}
\end{equation}
gdzie suma obejmuje sąsiednie spiny $s_j$, czyli te, które znajdują się bezpośrednio obok rozważanego spinu w siatce. Zasady akceptacji są następujące:
\begin{itemize}
    \item jeśli $\Delta E \le 0$, zmiana jest zawsze akceptowana (układ traci energię, którą przejmuje demon),
    \item jeśli $\Delta E > 0$ i demon ma energię $E_d \ge \Delta E$, zmiana jest akceptowana, a demon traci energię $\Delta E$,
    \item w przeciwnym przypadku zmiana jest odrzucana.
\end{itemize}
W kolejnych częściach pracy zaprezentowano wyniki symulacji, które pozwalają prześledzić zmiany temperatury oraz magnetyzacji, a także wskazać moment zaniku uporządkowania. 

Przedstawiono wyniki : histogram energii demona wykorzystany do dopasowania liniowego, przebieg średniej magnetyzacji w kolejnych krokach Monte Carlo oraz zależność średniej magnetyzacji od temperatury.

\section{Rozwinięcie}
\subsection{Środowisko i konfiguracja programu}

Implementację programu wykonano w środowisku \texttt{CLion 2024.3.4} (JetBrains).
Kod źródłowy napisano w standardzie \texttt{C++17} i kompilowano przy użyciu kompilatora \texttt{gcc}.\\[0.2cm]
Wykorzystano biblioteki:\\
\texttt{<vector>}, \texttt{<map>}, \texttt{<random>}, \texttt{<cmath>}, \texttt{<fstream>}, \texttt{<iostream>}, \texttt{<string>} oraz \texttt{<algorithm>}. \\ [0.2cm]
Wybór standardu \texttt{C++17} umożliwił wykorzystanie biblioteki \texttt{<filesystem>} do automatycznego tworzenia katalogów do zapisu danych.\\[0.2cm]
Program został przystosowany do uruchamiania z poziomu terminala.\\[0.2cm]
Dane wejściowe wczytywano z pliku \texttt{dane.txt}, który zawierał parametry symulacji: rozmiary siatki \((L_x, L_y)\), liczbę kroków Monte Carlo \(n_\text{steps}\), liczbę początkowych energii demona oraz zestaw ich wartości \(E_d\).\\[0.2cm]
Podczas działania programu dane były zapisywane automatycznie:
\begin{itemize}
    \item co 200 kroków Monte Carlo zapisywano histogram energii demona i przebieg magnetyzacji do osobnych plików:
    \begin{itemize}
        \item \texttt{histogram/histogram\_E=...txt} – histogram energii demona zawierający wartości $E$, liczby wystąpień $N(E)$ oraz $\ln N(E)$, wykorzystywane do wyznaczania temperatury,
        \item \texttt{magnetization/magnetization\_E=...txt} – chwilowa magnetyzacja $m(t)$ w~kolejnych krokach symulacji.
    \end{itemize}
    \item po zakończeniu symulacji dla każdej wartości energii demona wyniki zbiorcze (średnia magnetyzacja, nachylenie i obliczona temperatura) zapisywano do pliku:
    \begin{itemize}
        \item \texttt{mT.txt}.
    \end{itemize}
\end{itemize}
Dzięki takiemu sposobowi zapisu możliwa była analiza przebiegu magnetyzacji w czasie, histogramu energii demona oraz zależności $\langle m \rangle (T)$ po zakończeniu całej serii symulacji.

\subsection{Przetwarzanie danych symulacyjnych}
Podczas symulacji monitorowano:
\begin{itemize}
    \item zmianę energii demona w czasie (transfery energii między demonem a układem),
    \item zmiany magnetyzacji układu w kolejnych krokach Monte Carlo,
    \item średnią magnetyzację $\langle m \rangle$ w funkcji temperatury $T$, wyznaczoną z rozkładu energii demona.
\end{itemize}
Średnią magnetyzację obliczano po fazie relaksacji, odrzucając pierwsze $20\%$ kroków jako niestabilne.
\newpage
\subsection{Wyniki}
\FloatBarrier
\subsubsection*{Siatka $37\times37$}
\noindent
\textbf{Parametry symulacji:}
\begin{description}
  \item[Liczba kroków Monte Carlo:] $n_\text{steps} = 10{,}000$
  \item[Zakres energii demona:] $E_\text{D} = 56 \text{ -- } 1520$
  \item[Krok energii demona:] $\Delta E_\text{D} = 8$
\end{description}

\vspace*{0.5cm}
\begin{figure}[H]
    \centering
    \includegraphics[width=0.7\linewidth]{histogramLin.png}
    \caption{Histogram w skali liniowej energii demona dla siatki $37\times37$ i $E_\text{D}=200$.}
    \label{fig:hist_37Lin}
\end{figure}
\vspace*{0.5cm}
\begin{figure}[H]
    \centering
    \includegraphics[width=0.7\linewidth]{HistogramLog.png}
    \caption{Histogram w skali logarytmicznej energii demona dla siatki $37\times37$ i $E_\text{D}=200$.}
    \label{fig:hist_37Log}
\end{figure}
\newpage
\vspace*{0.5cm}
\begin{figure}[H]
    \centering
    \includegraphics[width=1.0\linewidth]{magnetyzacjaALL.png}
    \caption{Cały przebieg średniej magnetyzacji $m(t)$ w kolejnych krokach Monte Carlo  dla siatki $37\times37$ $E_\text{D}=200$. }
    \label{fig:magnetizationall7}
\end{figure}
\vspace*{1cm}
\begin{figure}[H]
    \centering
    \includegraphics[width=1.0\linewidth]{magnetization1000.png}
    \caption{Fragment przebiegu średniej magnetyzacji $m(t)$ obejmujący 1000 kroków Monte Carlo $n_{\text{steps}}$ dla siatki $37\times37$ i $E_d = 200$.}
    \label{fig:magnetization1000}
\end{figure}
\vspace*{1 cm}
\begin{figure}[H]
    \centering
    \includegraphics[width=1.0\linewidth]{mT.png}
    \caption{Zależność średniej magnetyzacji $\langle m \rangle$ od temperatury $T$ dla siatki $37\times37$.}
    \label{fig:mT_37}
\end{figure}

\FloatBarrier
\subsection{Komentarz do wyników}
Na podstawie uzyskanych wyników przeanalizowano zachowanie układu spinów w funkcji temperatury oraz wpływ rozmiaru siatki na kształt przejścia fazowego.

\paragraph{Histogram energii demona.\\}
Rozkład energii demona ma postać wykładniczą, zgodną z równaniem~(\ref{eq:demon_distribution}). Dane dopasowano do funkcji~\cite{ABRC2001} 
$N(E) = A e^{-aE}$,
co potwierdza poprawność działania algorytmu. Współczynnik \(A\) pełni rolę czynnika normalizacyjnego, natomiast parametr \(a\) odpowiada nachyleniu w skali logarytmicznej. Na podstawie histogramu energii demona \(N(E_d)\) przeprowadzono dopasowanie liniowe~(\ref{eq:regression}) metodą najmniejszych kwadratów~\cite{ABRC2001}.  
W tym celu obliczono sumy:
\begin{equation}
S_x=\sum_k x_k,\quad S_y=\sum_k y_k,\quad
S_{xx}=\sum_k x_k^2,\quad S_{xy}=\sum_k x_k y_k,
\label{eq:sums}
\end{equation}
gdzie \(x_k=E_k\), a \(y_k=\ln N(E_k)\).  
Na ich podstawie wyznaczono współczynniki regresji liniowej:
\begin{equation}
a = \frac{n S_{xy} - S_x S_y}{n S_{xx} - S_x^2}, \qquad
b = \frac{S_{xx} S_y - S_x S_{xy}}{n S_{xx} - S_x^2},
\label{eq:linreg}
\end{equation}
gdzie \(n\) oznacza liczbę punktów histogramu.\\[0.2cm]
Uzyskane współczynniki posłużyły następnie do wyznaczenia temperatury~(\ref{eq:temperature}). \\[0.2cm]
Niewielkie odchylenia od wartości teoretycznych wynikają z ograniczonej liczby próbek oraz statystycznego charakteru metody Monte Carlo.
\paragraph{Ewolucja magnetyzacji w czasie.\\}
Po początkowym etapie relaksacji układ osiąga stan równowagi, w którym magnetyzacja oscyluje wokół stałej wartości.
Niewielkie, chwilowe zmiany magnetyzacji wynikają z losowych zmian orientacji pojedynczych spinów, które pojawiają się w trakcie symulacji. Nie wpływają one jednak znacząco na ogólny stan uporządkowania układu.
\paragraph{Zależność średniej magnetyzacji od temperatury.\\}
Jednym z kluczowych wyników przeprowadzonej symulacji jest zależność średniej magnetyzacji \(\langle m\rangle\) od temperatury \(T\). Na wykresie~(\ref{fig:mT_37}) widać wyraźny spadek magnetyzacji wraz ze wzrostem temperatury oraz jej wahania w pobliżu temperatury krytycznej. Te wahania wynikają z losowego charakteru metody Monte Carlo: kolejne konfiguracje układu powstają na podstawie decyzji probabilistycznych, dlatego w obszarze przejścia fazowego obserwujemy dużą zmienność wyników. Punkty na wykresie układają się wtedy w charakterystyczną „mgłę” przy przejściu fazowym — oznacza to, że układ łatwo przechodzi między różnymi konfiguracjami spinów, a stany generowane przez algorytm stają się wzajemnie powiązane. \\\\\\
Teoretyczna temperatura krytyczna dla modelu Isinga 2D wynosi $T_c^{\mathrm{teor}}=2.269\,\frac{J}{k_B}$. 
Najbliższą tej wartości temperaturę uzyskano dla 
\(T = 2.3237\,\frac{J}{k_B}\) (początkowa energia demona \(E_d = 1112\)); 
w tym punkcie \(\langle m\rangle \approx 4.53\times10^{-2}\), 
co odpowiada niewielkiej, dodatniej magnetyzacji bliskiej zeru.
Względna różnica względem wartości teoretycznej wynosi
\[
\delta = \frac{|2.3237 - 2.269|}{2.269}\times 100\% \approx 2.4\%,
\]
co jest dość dobrym wynikiem jak na symulacje Monte Carlo przeprowadzone na ograniczonych małych rozmiarach siatki. 
W małych układach przejście fazowe bywa „rozmyte” i może przesunąć się względem wartości teoretycznej, 
stąd nieznaczne odchylenie jest spodziewane.\\[1.cm]


\section{Zakończenie}
Uzyskane wyniki wykazują oczekiwany przebieg charakterystyczny dla przejścia fazowego — wzrost temperatury prowadzi do stopniowego zaniku uporządkowania i gwałtownego spadku magnetyzacji w pobliżu temperatury krytycznej.\\[0.2cm]
Wynik symulacyjny \(T_c^{\mathrm{sym}} \approx 2.3237\,\frac{J}{k_B}\) (dla \(E_d = 1112\)) jest bliski wartości teoretycznej \(T_c^{\mathrm{teor}} = 2.269\,\frac{J}{k_B}\), a różnica \(\delta \approx 2.4\%\) mieści się w akceptowalnym zakresie błędu. 
Drobne odchylenia można przypisać ograniczonemu rozmiarowi siatki oraz losowemu charakterowi procesu Monte Carlo, który w pobliżu temperatury krytycznej prowadzi do wahań magnetyzacji i niewielkich przesunięć obserwowanej temperatury przejścia.


\newpage
\bibliographystyle{plain}
\bibliography{bibliography}

\end{document}

